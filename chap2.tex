\section{Turing Machine and RAM}

\subsection{Turing Machine}

\key{Turing Machine} $M=\langle
Q,\Sigma,\Gamma,\delta,q_0,B,q_{accept},q_{reject}\rangle$

\key{TM-acceptable} A language $L, L \subseteq \Sigma^*$, is
Turing-machine-acceptable if there is a Turing machine
that accepts L.

\key{TM-decidable} A language L is Turing-machine-decidable if L is accepted by
some
Turing machine that halts on every input.

\key{$L(M)$} $L(M)=\{w \in \Sigma^* | M \text{ accepts } w \}$.

\key{$M$ computes $\phi$} $M$ eventually enter an accepting configuration of
$\phi(w_1,\ldots,w_n)q_{accept}$ iff.
$\phi(w_1,\ldots,w_n)\downarrow$.

\key{Partial Computable} A partial function $\phi$ is partial computable if there
is some Turing machine
that computes it.

\key{Total Computable} $\phi(w_1,\ldots,w_n)\downarrow$ for all
$w_1,\ldots,w_n$. If $M$ computes a total computable function, $L(M)=\Sigma^*$.

\key{Theorem 2.2.} A language $L$ is decidable if and only if both $L$ and
$\bar{L}$ are acceptable.\\
Parallel simulation.

\key{RAM}
\begin{table}[H]
  \begin{tabular}{lll}
    $1_j$ & $X\;\mathbf{add}_j\;Y$ & append $a_j$ to $Y$\\
    $2$ & $X\;\mathbf{del}\;Y$ & delete right most symbol of $Y$\\
    $3$ & $X\;\mathbf{clr}\;Y$ & $Y \rightarrow \lambda$\\
    $4$ & $X\;Y\leftarrow Z$ & $Y=Z$\\
    $5$ & $X\;\mathbf{jmp}\;Y$\\
    $6_j$ & $X\;Y\;\mathbf{jmp}_j\;X'$\\
    $7$ & $X\;\mathbf{continue}$
  \end{tabular}
\end{table}


