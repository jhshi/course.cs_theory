\section{Turing Machine and RAM}

\subsection{Turing Machine}

\key{Turing Machine} $M=\langle
Q,\Sigma,\Gamma,\delta,q_0,B,q_{accept},q_{reject}\rangle$

\subsection{Turing Machine Concepts}

\key{Definition 2.1.} Let $M$ be a Turing machine. The language accepted by $M$ is
$L(M) = \{w \in \Sigma^* | M \text{ accepts } w\}$.

\key{TM-acceptable} A language $L, L \subseteq \Sigma^*$, is
Turing-machine-acceptable if there is a Turing machine
that accepts L.

\key{Definition 2.2.} A language $L$ is Turing-machine-decidable if $L$ is accepted
by some Turing machine that halts on every input, and a Turing machine that halts on
every input and accepts $L$ is called a decider for $L$.

\key{$M$ computes $\phi$} $M$ eventually enter an accepting configuration of
$\phi(w_1,\ldots,w_n)q_{accept}$ iff.
$\phi(w_1,\ldots,w_n)\downarrow$.

\key{Partial Computable} A partial function $\phi$ is partial computable if there
is some Turing machine
that computes it.

\key{Total Computable} $\phi(w_1,\ldots,w_n)\downarrow$ for all
$w_1,\ldots,w_n$. If $M$ computes a total computable function, $L(M)=\Sigma^*$.

\key{Theorem 2.2.} A language $L$ is decidable if and only if both $L$ and
$\overline{L}$ are acceptable.\\
Parallel simulation.


\key{Proposition 2.1.} If $L$ is decidable, then $\overline{L}$ is decidable.

\subsection{Variations of Turing Machines}

\subsubsection{Multitape Turing Machines}

\key{Definition 2.3.} Two Turing machines are equivalent if they accept the same
language.

\key{Theorem 2.1.} Every multitape Turing machine has an equivalent one-tape
Turing machine.

\key{Corollary 2.1.} For every multitape Turing machine there is a one-tape
Turing machine that computes the same partial computable function.

\key{Theorem 2.2.} A language $L$ is decidable if and only if both $L$ and
$\overline{L}$ are acceptable.

\subsubsection{Nondeterministic Turing Machines}

\key{Theorem 2.3.} Every nondeterministic Turing machine has an equivalent
deterministic Turing machine.

\key{Corollary 2.2.} If every computation path of a nondeterministic Turing
machine $N$ halts on every input word, then there is a deterministic Turing
machine $M$ that decides the language $L(N)$.

\subsection{Church's Thesis}

\subsection{RAMs}

\key{RAM}
\begin{table}[H]
  \begin{tabular}{lll}
    $1_j$ & $X\;\mathbf{add}_j\;Y$ & append $a_j$ to $Y$\\
    $2$ & $X\;\mathbf{del}\;Y$ & delete right most symbol of $Y$\\
    $3$ & $X\;\mathbf{clr}\;Y$ & $Y \rightarrow \lambda$\\
    $4$ & $X\;Y\leftarrow Z$ & $Y=Z$\\
    $5$ & $X\;\mathbf{jmp}\;Y$\\
    $6_j$ & $X\;Y\;\mathbf{jmp}_j\;X'$\\
    $7$ & $X\;\mathbf{continue}$
  \end{tabular}
\end{table}

\key{Theorem 2.4.} Every RAM program can be effectively transformed into an
equivalent one that uses only instructions of types 1, 2, 6, and 7.

\subsubsection{Turing Machines for RAMS}
