\section{Nondeterminism and NP-Completeness}

\subsection{Characterizing NP}

\key{Theorem 6.1.} A set $A$ belongs to NP if and only if there exist a polynomial
$p$ and a binary relation $R$ that is decidable in polynomial time such that for all words
in $\Sigma^*$,
$x \in A \Leftrightarrow \exists y [|y| \le p(|x|) \land R(x, y)]$.

\key{Verifier} Define a verifier for a language $A$ to be an algorithm $V$ such that
$A = \{x |\exists y[V \text{ accepts } \langle x, y \rangle ]\}$.

\key{Corollary 6.1.} NP is the class of all languages A having a polynomial-time
verifier.

\subsection{The Class P}

\subsection{Enumerations}

\key{Definition 6.1.} A class of sets $\mathscr{C}$ is effectively presentable if there is an
effective enumeration $\{M_i\}_i$ of Turing machines such that every Turing machine in the
enumeration halts on all inputs and $\mathscr{C} = \{L(M_i) | i \ge 0\}$.

\key{Theorem 6.2.} There is no effective enumeration of the class of all
deterministic Turing machines that operate in polynomial time. That is,
$S = \{i | \text{DM}_i \text{operates in polynomial time}\}$
is not a computably enumerable set.

\key{Theorem 6.3.} P and NP are effectively presentable:\\
NP = \{L(NP$_i) | i \ge 0\}$;\\
P = \{L(P$_i) | i \ge 0\}$;

\subsection{NP-Completeness}


