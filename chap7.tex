\section{Relative Computability}

\key{$L(M,A)$} The language accepted by $M$ with oracle $A$ is denoted $L(M, A)$

\key{Definition 7.1.} A set $A$ is Turing-reducible to $B$ in polynomial-time 
($A \le_T^P B$) if there exists a deterministic polynomial-time-bounded oracle 
Turing machine M such that $A = L(M, B)$.

\key{Theorem 7.1.} There is a decidable set $A$ such that $\overline{A} \le_m^P A$ (and
$A \ne \Sigma^*$ and $A \ne \emptyset$).

\subsection{NP-Hardness}

\key{Definition 7.2.} A set $A$ is NP-hard if, for every $L \in$ NP,
$L \le^P_T A$.
\begin{itemize}
  \item An NP-hard set does not need to belong to NP
  \item We use Turing reducibility this time instead of many-one reducibility
  \item Every NP-complete set is NP-hard
  \item The complement of every NP-complete set is NP-hard
\end{itemize}

\key{Proposition 7.1.} If $A$ is NP-hard and $A \in $P, then NP = P.

\key{Theorem 7.3}. For each decidable set $A \notin $ P, there is a
decidable set $B$ such that $A \le^P_T B$ but $A \nleq^P_m B$. In particular, 
$A \le^P_T B$ by a reduction procedure that on every input makes two queries to the oracle.

\key{Corollary 7.1.} If P $\ne$ NP, then there exists a set that is $\le^P_T$-hard 
for NP but not $\le^P_m$-hard for NP.

\key{Theorem 7.4.} If A is $\le^P_T$-complete for NP, then $A \in P$ if and only if P = NP.

\subsection{Search Problems}

\key{Definition 7.3.} Let $L \in NP$ and let $R_L$ and $p_L$ define $L$.
Prefix($R_L, p_L$) = $\{\langle x, u \rangle | u$ is a prefix of a witness $y$ such that
$|y| \le p_L(|x|)$ and $R_L(x, y)$\}.

\key{Proposition 7.2.}
\begin{enumerate}
  \item Prefix($R_L,p_ L) \in$ NP.
  \item $L \le^P_m$ Prefix($R_L,p_ L$).
  \item If $L$ is NP-complete, then Prefix($R_L,p_L$) is NP-complete.
  \item If $L$ is $\le^P_T$-complete for NP, then Prefix($R_L,p_ L$) is
    $\le^P_T$-complete for NP.
\end{enumerate}

\key{Theorem 7.5.} The search problem for $R_L$ and $p_L$ is Turing-reducible in
polynomial time to Prefix($R_L,p_L$).

\subsection{The Structure of NP}

\key{Definition 7.6.} Two sets $A$ and $B$ are equal almost everywhere ($A = B$ a.e.)
if the symmetric difference of $A$ and $B$, $A \vartriangle B$, is a finite set.
A class of sets $\mathscr{C}$ is closed under finite variations if $A \in
\mathscr{C}$ and $A = B$ a.e. implies $B \in \mathscr{C}$.

The complexity classes P and NP are closed under finite variation.

\key{Fast function} Define a function $f : N \rightarrow N$ to be fast if
the following two properties hold:
\begin{enumerate}
  \item For all $n \in N$, $f(n) > n$, and
  \item There is a Turing machine $M$ that computes $f$ in unary notation such
    that $M$ writes a symbol on its output tape every move of its computation. In particular,
    for every $n$, $f(n) = T_M(n)$.
\end{enumerate}

\key{Proposition 7.3.} For every total computable function $f$ , there is a fast
function $f'$ such that, for all $n$, $f'(n) > f(n)$.

\key{$\mathbf{G[f]}$} $G[f] = \{ x \in \Sigma^* | f^n(0) \le |x| < f^{n+1}(0)$, for even
$n$\}.

\key{Lemma 7.1.} If $f$ is fast, then $G[f] \in P$.


\key{Theorem 7.6.} Let $A$ and $B$ be decidable sets and let $\mathscr{C}_1$ and
$\mathscr{C}_2$ be classes of decidable sets with the following properties:
\begin{enumerate}
  \item $A \notin \mathscr{C}_1$ and $B \notin \mathscr{C}_2$;
  \item $\mathscr{C}_1$ and $\mathscr{C}_2$ are effectively presentable; and
  \item $\mathscr{C}_1$ and $\mathscr{C}_2$ are closed under finite variations.
\end{enumerate}
Then there exists a decidable set C such that
\begin{enumerate}
  \item $C \notin \mathscr{C}_1$ and $C \notin \mathscr{C}_2$, and
  \item If $A \in$ P and $B \ne \emptyset$ and $B \ne \Sigma^*$, then $C
    \le^P_m B$.
\end{enumerate}

\key{Lemma 7.2.} The class of all $\le^P_T$-complete sets for NP is effectively presentable.

\key{Corollary 7.4.} If P $\ne$ NP, then there exists a set $C$ in NP-P
that is not $\le^P_T$-complete for NP.\\
Let $A=\emptyset$, $B=$SAT, and let $\mathscr{C}_1$ be collection of all
$\le^P_T$-complete sets of NP, and $\mathscr{C}_2 =$P.
\begin{enumerate}
  \item $C \notin$ P and $C$ is not $\le^P_T$-complete for NP
  \item $C \le^P_m$ SAT, so $C \in$ NP.
\end{enumerate}

\key{$\mathbf{X \oplus Y}$} $X \oplus Y = \{0x | x \in X\} \bigcup \{1x | x \in
Y\}$.

\key{Corollary 7.5.} If P $\ne$ NP, then there exist $\le^P_T$-incomparable
members of NP. That is, there exist sets $C_0$ and $C_1$ in NP such that 
$C_0 \nleq^P_T C_1$ and $C_1 \nleq^P_T C_0$ .

\key{Corollary 7.6.} If P $\ne$ NP, then for every set $B \in$ NP-P, there is a 
set $C \in$ NP-P such that $C \le^P_T B$ and $B \nleq^P_T C$.\\
If P $\ne$ NP, then NP contains countably many distinct $\le^P_T$-degrees that form 
an infinite descending hierarchy.

\key{Corollary 7.7.} If P $\ne$ NP, then NP$-$P is not effectively presentable.

\subsubsection{Composite Number and Graph Isomorphism}

\subsection{The Polynomial Hierarchy}

For any set $A$, let P$^A = \{B | B \le^P_T A\}$ and let NP$^A = \{B |
B\le^{NP}_T A\}$. So P$^A$(NP$^A$) is the class of sets accepted deterministically
(nondeterministically, respectively) in polynomial time relative to the set $A$.

\key{Polynomial hierarchy} $\Sigma^P_0 = \Pi^P_0 = \Delta^P_0 =$ P
\begin{align*}
  \Sigma^P_{k+1} & = \text{NP}^{\Sigma^P_{k}} \\
  \Pi^P_{k+1} &= \text{co-}\Sigma^P_{k+1} \\
  \Delta^P_{k+1} &= \text{P}^{\Sigma^P_{k}} 
\end{align*}

\key{Proposition 7.4.} For all $k \ge 0$, $\Sigma_k^P \bigcup \Pi_k^P \subseteq
\Delta_{k+1}^P \subseteq \Sigma_{k+1}^P \bigcap \Pi_{k+1}^P$.

\key{Proposition 7.5.} PH $\subseteq$ PSPACE, where PH= $\bigcup \{\Sigma^P_k|k \ge
0\}$.

\key{Theorem 7.10.} If for some $k \ge 1, \Sigma_k^ P = \Pi_k^P$ , then for all
$j \ge k$, $\Sigma^P_j = \Pi^P_j = \Sigma_k^P$ .

\key{Corollary 7.8.} $A \le^P_m B$ and $B \in \Sigma�_n^P$ implies $A \in
\Sigma_n^P$.
