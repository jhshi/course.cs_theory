\section{Undecidability}

\subsection{Decision Problems}

\subsection{Undecidable Problems}

\key{Characteristic Function} 
$f_{S}(x) = \begin{cases}
  0 & \text{if } x \in S \\ 
  1 & \text{if } x \notin S
\end{cases}$

\key{Proposition 3.1.} A set $S$ is decidable if and only if its characteristic
function is computable.

\key{$w_M$} The word that encodes $M$.

\key{G\"odel Number $e$} The code for a Turing machine $M$.

\key{$\phi_e=\lambda x.U(e,x).$} $\phi_e$ is the partial function of one
argument that is computed by $M_e$.

\key{Theorem 3.1.} The Program Termination problem (Example 3.2) is undecidable.
There is no algorithm to determine whether an arbitrary partial computable
function is total. Thus, there is no algorithm to determine whether a Turing
machine
halts on every input.\\
Define
\begin{align*}
  \text{TEST}(i) = 
  \begin{cases}
    \text{``yes''} & \text{ if } \phi_i \text{ halts on every input.} \\
    \text{``no''} & \text{ otherwise}
  \end{cases}
\end{align*}

\begin{align*}
  \delta(k) =
  \begin{cases}
    \phi_k(k)+1 & \text{ if TEST(i)="yes"} \\
    0 & \text{ if TEST(k)="no"}
  \end{cases}
\end{align*}
Thus $\delta$ is total computable. Let $\delta=\phi_e$, then TEST(e) is ``yes''.
$\delta(e)=\phi_e(e)+1 \neq \phi_e(e)$.

\subsection{Pairing Functions}
\key{Paring function} Computable one-to-one mapping $<,>:N \times N \rightarrow
N$, whose inverse $\tau_1(<x,y>)=x$ and $\tau_2(<x,y>)=y$ are also computable.\\
Example, $<x,y>=\frac{1}{2}(x^2+2xy+y^2+3x+1)$.


\subsection{Computably Enumerable Sets}
\key{Computable enumerable (c.e.)} A set $S$ is c.e. if $S=\emptyset$ or
$S=range(f)$ in which $f$ is a total computable function.

\key{index set} Let $\mathscr{C}$ be any set of partial computable functions,
then $P(\mathscr{C})=\{e|\phi_e \in \mathscr{C}\}$ is called index set.

\key{Effectively enumerable} A collection of Turing machines is effectively
enumerable if the corresponding set of G\"odel numbers is c.e.

\key{Homework 3.3} $A=\{(e,j)|L(M_e) = L(M_j)\}$ is not decidable.\\
Suppose by contradiction that $A$ is decidable, let $L(M_j)=\Sigma^*$, then
$\{e|L(M_e)=\Sigma^*\}=\{e|\phi_e \text{ is total computable}\}$ is decidable,
this contradicts \textbf{Theorem 3.1}.

\key{Theorem 3.2.} $\{e | \phi_e is \text{ total computable}\}$ is not
computably enumerable.\\
Let $S=\{e|\phi_e \text{ is total computable}\}$ and suppose for contradiction
that $S$ is c.e., then $S=range(g)$ for some total computable function $g$.
Define $U_S(e,x)=\phi_{g(e)}(x)$ and $h(x)=U_S(x,x)+1$, so $\exists k \in S
\text{ s.t. } h=\phi_k$ and $\exists e \text{ s.t. } k=g(e)$.
Finally,
\begin{align*}
  \phi_k(e)&=h(e)\\
  &=U_S(e,e)+1 \\
  &=\phi_{g(e)}(e)+1\\
  &=\phi_k(e)+1
\end{align*}

\key{Theorem 3.3.} A set $S$ is computably enumerable if and only if there is a
decidable
relation $R(x, y)$ such that
\begin{align*}
x \in S \Leftrightarrow \exists y R(x, y).
\end{align*}

\key{Theorem 3.4.} A set $S$ is computably enumerable if and only if it is
Turing-machine-
acceptable.

\key{Corollary 3.1.} A set $S$ is decidable if and only if $S$ and $\overline{S}$ are both
computably
enumerable.

\key{Corollary 3.2.} A set $S$ is computably enumerable if and only if $S$ is the
domain of
some partial computable function.

\key{Homework 3.4} Prove that an infinite set is decidable if and only if it can
be
enumerated in increasing order by a one-to-one total computable function.

\key{Homework 3.6} Prove that every infinite c.e. set contains an infinite
decidable subset.
\begin{align*}
  h(0) &= f(0)\\
  h(n+1) &= f(min\{x|f(x) \notin \{h(0), h(1),\ldots, h(n)\}\})
\end{align*}

\key{$W_e=dom(\phi_e)$}

\subsection{Halting Problem, Reductions, and Complete Sets}

\key{Diagonal Set} $K=\{x|\phi_x(x)\downarrow\}=\{x|U(x,x)\downarrow\}$=\{TM
that accepts its own code.\}. Since $\lambda x.U(x,x)$ is partial computable,
$K$ is c.e.

\key{Theorem 3.5.} $K$ is not decidable. In particular, $\overline{K}$ is not
c.e.\\
Suppose $\overline{K} = W_e = dom(\phi_e)$, then $e \in \overline{K} \Leftrightarrow
\phi_e(e)\downarrow \Leftrightarrow e \in K$.

\key{Many-one reducible} $A \le_m B$ if there is a total computable function
s.t. $x \in A \Leftrightarrow f(x) \in B$

\key{Lemma 3.2.} 1. If $A \le_m B$ and $B$ is c.e., then $A$ is c.e.\\
2. If $A \le_m B$ and $B$ is decidable, then $A$ is decidable.

\key{Theorem 3.6.} The Halting problem is undecidable. Specifically, the set
$L_U=\{(e,w)|M_e \text{ accepts } w\}$ is not decidable.\\
$x \in K \Leftrightarrow (x,x) \in L_U$, $x \mapsto (x,x)$ is total, so $K \le_m
L_U$.

\subsubsection{Complete Problems}
\key{Many-one complete} $L$ is many-one complete if
\begin{enumerate}
  \item $L$ is c.e.
  \item For every c.e. set $A$, $A \le_m L$
\end{enumerate}

\key{Homework 3.8} Show that $K$ is a many-one complete set. Note that it suffices
to show that $L_U \le_m K$.\\
Need to show $(e,w) \in L_U \Leftrightarrow f((e,w)) \in K$ for some total
computable function $f$. Define $f((e,w)) = e'$ where $M_e'$ is defined as
follows.\\
\begin{algorithm}[H]
  on input x\;
  \uIf{$M_e$ accepts $w$}{
    ACCEPT\;
  }
  \Else{
    REJECT\;
  }
\end{algorithm}
Then we have
\begin{align*}
  (e,w) \in L_U & \Leftrightarrow L(M_e')=\Sigma^*\\
  & \Leftrightarrow e' \in L(M_e')\\
  & \Leftrightarrow e' \in K
\end{align*}

Decidable sets are a proper subclass of the set of all c.e. sets

\subsection{S-m-n Theorem}

\key{Corollary 3.3.} For every partial computable function $\lambda x.\Psi(e,x)$,
there is a total computable function $f$ so that $\phi_{f(e)}(x) = \Psi(e,x).$

\key{Theorem 3.9.} There is a total computable function $f$ such that
$range \phi_{f(e)} = dom \phi_e$.\\
Define 
\begin{align*}
  \Psi(e,x)=
  \begin{cases}
    x & \text{ if } x \in dom\phi_e\\
    \uparrow & \text{ otherwise.}
  \end{cases}
\end{align*}
So $range(\lambda x.\Psi(e,x)) = dom\phi_e$, and $\phi_{f(e)}=\Psi(e,x)$, so
$range(\phi_{f(e)}) = dom\phi_e$.

\key{Homework 3.9} Prove that there is a total computable function $g$ such that
$dom\phi_{g(e)} = range\phi_e$.\\
Define
\begin{align*}
  \Psi(e,x)=
  \begin{cases}
    1 & \text{ if } x \in range\phi_e\\
    \uparrow & \text{ otherwise.}
  \end{cases}
\end{align*}
So $range(\Psi(e,x)) = range(\phi_{g(e)}(x) = range(\phi_e)$

\subsection{Recursion Theorem}

\key{Theorem 3.10.} For every total computable function $f$ there is a number
$n$
such that $\phi_n = \phi_{f(n)}$. A number $n$ with this property is called a
fixed point of $f$ .

\key{Corollary 3.5.} There is a number (i.e., program) $n$ such that $\phi_n$ is the
constant function with output $n$.\\
Define $\Psi(e,x)=e$, then $\Psi(e,x)=\phi_{f(e)}(x)=\phi_e(x) = e$.

\key{$W_n=\{n\}$} Define
\begin{align*}
  \Psi(e,x)=
  \begin{cases}
    e & \text{ if } x = e \\
    \uparrow & \text{ otherwise}
  \end{cases}
\end{align*}
$\Psi(e,x)=\phi_{f(e)}(x)=\phi_e(x)$, $dom\phi_e = \{e\}$

\key{$W_n=\{n^2\}$} Define
\begin{align*}
  \Psi(e,x)=
  \begin{cases}
    e & \text{ if } x = e^2 \\
    \uparrow & \text{ otherwise}
  \end{cases}
\end{align*}
$\Psi(e,x)=\phi_{f(e)}(x)=\phi_e(x)$, $dom\phi_e = \{e^2\}$

\key{Homework 3.11} Show that there is no algorithm that given as input a Turing 
machine $M$, where $M$ defines a partial function of one variable, outputs
a Turing machine $M'$ such that $M'$ defines a different partial function of one
variable.\\
Suppose for contradiction that such $\exists f \forall n \phi_n \ne
\phi_{f(n)}$, and $f$ is total.

\key{Theorem 3.11.} For every partial computable function $\Psi(e, x)$, there is a
value $e_0$ such that $\Psi(e_0,x) = \phi_{e_0}(x)$.

Observe that there is a standard pattern to the proof of these results. First,
we use the $s-m-n$ theorem or its corollary to obtain a total computable
function $f$ with whatever property we find useful. Then, we use the recursion theorem or its
corollary to select a fixed point of $f$.


\subsection{Rice's Theorem}

\key{Theorem 3.12.} An index set $P_{\mathscr{C}}$ is decidable if and only if
$P_{\mathscr{C}} = \emptyset$ or $P_{\mathscr{C}}= N$.\\
Suppose $P_{\mathscr{C}} \neq \emptyset$ and $P_{\mathscr{C}} \neq N$, let 
$j \in P_{\mathscr{C}}$ and $k \notin P_{\mathscr{C}}$, define
\begin{align*}
  f(x) = 
  \begin{cases}
    k & \text{ if } x \in P_{\mathscr{C}} \\
    j & \text{ if } x \notin P_{\mathscr{C}}
  \end{cases}
\end{align*}
Suppose for contradiction that $P_{\mathscr{C}}$ is decidable, then $f$ is
total, then $f$ has a fixed point $n$ such that $\phi_n = \phi_{f(n)}$. Since $n$
and $f(n)$ is the code for same partial functions, either they both belong to
$P_{\mathscr{C}}$ or both belong to $\overline{P_{\mathscr{C}}}$, but 
$x \in P_{\mathscr{C}} \Leftrightarrow f(x) \notin P_{\mathscr{C}}$.

To use Rice's theorem to show that a set $A$ is not decidable, the set
$A$ must be an index set. Therefore, if one program $e$ to compute $\phi_e$
belongs to $A$, then every
program $i$ such that $\phi_i = \phi_e$ must also belong to $A$. Thus,
Rice's theorem only applies to machine-independent properties.

\subsection{Turing Reductions and Oracle Turing Machines}

\key{$M^A$} an oracle TM with $A$ as its oracle.

\key{Definition 3.5.} $A$ is decidable in $B$ if $A = L(M^B)$, where $M^B$ halts on
every input.

\key{Definition 3.6.} $A$ is Turing-reducible to B if and only if A is decidable in
B. In notation: $A \le_T B$.

\key{Homework 3.13} Prove each of the following properties:\\
1. $\le_T$ is transitive;\\
2. $\le_T$ is reflexive;\\
3. For all sets $A$, $A \le_T \overline{A}$;\\
4. If $B$ is decidable and $A \le_T B$, then $A$ is decidable; \\
5. If $A$ is decidable, then $A \le_T B$ for all sets $B$;\\
6. $A \le_m B \Rightarrow A \le_T B$;\\
7. $\exists A, B[A \le_T B \text{ and } A \nleq_m B]$;\\
$\overline{K} \le_T K$ but $\overline{K} \nleq_m K$.\\
8. $\exists A, B[A \le_T B$ and $B$ is c.e. and $A$ is not c.e.].
$\overline{K} \le_T K$, $K$ is c.e., $\overline{K}$ is not c.e.


