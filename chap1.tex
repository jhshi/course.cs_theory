\section{Preliminaries}

\subsection{Words and Language}

\key{Alphabet} A finite set of symbols. $\Sigma=\{a_1, a_2,\ldots,a_k\}$.

\key{Word} A finite sequence of symbols.

\key{Language} A set of words. $L \in \Sigma^*$.

\subsection{Partial Functions}

\key {Partial Function} $f: X' \rightarrow Y, \text{where } X' \subset X$.

\key{Total Function} When $X' = X$.

\key{Converge} When $f(x)$ is defined. Noted as $f(x) \downarrow$.

\key{Diverge} When $f(x)$ is not defined. Noted as $f(x) \uparrow$.

\subsection{Propositional Logic}

\key{Satisfiable}  A formula $F$ is
satisfiable if there exists an assignment to its variables that satisfies it.

\key{Tautology} A formula is valid (or is a tautology) if every assignment to
its
variables satisfies it

\key{Conjunction} $A_1 \land A_2 \land \cdots \land A_n$

\key{Disjunction} $A_1 \lor A_2 \lor \cdots \lor A_n$

\key{Clause} Disjunction of literals.

\key{Conjunctive Normal Form (CNF)} Conjunction of clauses.

\subsection{cardinality}

\key{Same Cardinality} $card(A) = card(B) \text{ iff. } \exists f: A \rightarrow
B$ is a bijection.

\key{Countable} A set $A$ is countable if $card(A) = card(N)$
or $A$ is finite.

\key{Countable Infinite} $card(A) = card(N)$.

\key{Enumerable} A set is enumerable if it is the empty set or there is a
function $f : N \rightarrow_{onto} A$, i.e., $A = range(f) = \{a_0, a_1, \ldots\}$

\key{Enumerable $\Rightarrow$ Countable} Define $h$ as follows:
\begin{align*}
  h(0) &= f(0) \\
  h(n+1) &= f(min\{x|f(x) \notin \{h(0), h(1),\ldots, h(n)\}\})
\end{align*}
\begin{itemize}
  \item $h$ is one-to-one since $h(n+1) \notin \{h(0), h(1),\ldots,h(n)\}$. \\
  \item $range(h) \subseteq range(f) = S$. \\
  \item $f(0)=h(0)$, suppose by induction that $f(n) \in \{h(0),h(1),\ldots,h(n)\}$ and $f(n+1) \notin
    \{h(0), h(1),\ldots,h(n)\}$, then $n+1 = min\{x|f(x) \notin
    \{h(0),h(1),\ldots,h(n)\}\}$, so $h(n+1)=f(n+1)$. So $\forall n, f(n) \in \{h(0),
    h(1),\ldots,h(n)\}$, $S=range(f) \subseteq range(h)$.\\
\end{itemize}
Thus $S=range(f) = range(h)$, $h:N\rightarrow_{1-1}S$, $card(N)=card(S)$

\key{Theorem 1.3.} A set $A$ is countable if and only if $card(A) \le
\mathbf{\aleph}_0$.
\begin{align*}
& card(A) \le \aleph_0 \\
\Rightarrow & \exists f:A \rightarrow_{1-1} N \\
\Rightarrow & f[A] \text{ doesn't have a largest number (otherwise, } A \text{ is
finite.)}\\
\Rightarrow & a_0 = min\{f[A]\}, a_{n+1}=min\{f[A]-\{f(0),f(1),\ldots,f(n)\}\}
\\
\Rightarrow & A \text{ is enumerable.} \\
\Rightarrow & A \text{ is countable.}
\end{align*}

\key{Theorem 1.4.} The set of all functions from $N$ to $N$ is not countable.\\
Let $A=\{f|f:N \rightarrow N\}$, suppose for contradiction that $A$ is countable,
then $A=\{f_1,f_2,\ldots\}$, define $g(x)=f_x(x)+1$ and $g = f_k$ for some $k$,
but $g(k)=f_k(k)+1 \neq f_k(k)$.

\key{Theorem 1.5.} $\mathscr{P}(N)$ has cardinality greater than $\aleph_0$.\\
Let $A=\mathscr{P}(N) = \{S|S \subseteq N\}$ is power set of $N$. Suppose for
contradiction that $A$ is enumerable, then $A=\{S_0,S_1,\ldots\}$, define
$T=\{k|k\notin S_k\}$ and $T \in A$. However, $\forall k\; T \neq S_k$ since $k
\in T \Leftrightarrow k \notin S_k$.

\subsection{Misc}

\key{onto/surjection} $f: A \rightarrow_{onto} B \text{ iff. } \forall b \in B, \exists a
\in A \text{ s.t. } f(a) = b$

\key{one-to-one/injection}  $f: A \rightarrow_{1-1} B \text{ iff. } f(a)=f(b)
\Rightarrow a=b$

\key{bijection} Both one-to-one and onto.


